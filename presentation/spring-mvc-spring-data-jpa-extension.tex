\section{Spring MVC et Spring Data JPA}
\begin{frame}
 \frametitle{Outline}
 \scriptsize{\tableofcontents[currentsection,currentsubsection]}
\end{frame}


\begin{frame}[fragile]
 \frametitle{Points d'extension de Spring MVC 3.1}
 \begin{itemize}
  \item Possibilité d'influer sur les paramètres de méthodes de contrôleurs
  \item Ex. : Comment est géré \code{@RequestBody} ?
 \end{itemize}
 \begin{javacode}
@RequestMapping(value="/contacts/{id}",method=RequestMethod.PUT)
@ResponseStatus(HttpStatus.NO_CONTENT)
public void update(@RequestBody Contact contact) {
  contactRepository.save(contact);
}
 \end{javacode}

\end{frame}

\begin{frame}[fragile]
 \frametitle{\code{HandlerMethodArgumentResolver}}
 \begin{javacode}
public interface HandlerMethodArgumentResolver {

  boolean supportsParameter(MethodParameter parameter);

  Object resolveArgument(MethodParameter parameter, 
     ModelAndViewContainer mavContainer,
     NativeWebRequest webRequest, 
     WebDataBinderFactory binderFactory) throws Exception;

}
 \end{javacode}

\end{frame}

\begin{frame}[fragile]
 \frametitle{\code{HandlerMethodArgumentResolver}}
 \begin{itemize}
  \item Principe pour la gestion de \code{@RequestBody}
 \end{itemize}
 \begin{javacode}
public class RequestBodyHandlerMethodArgumentResolver {

  public boolean supportsParameter(MethodParameter parameter) {
    return parameter.hasParameterAnnotation(RequestBody.class);
  }

  (...)

}
 \end{javacode}

\end{frame}

\begin{frame}[fragile]
 \frametitle{\code{HandlerMethodArgumentResolver}}
 \begin{itemize}
  \item Principe pour la gestion de \code{@RequestBody}
 \end{itemize}
 \begin{javacode}
public class RequestBodyHandlerMethodArgumentResolver {
  (...)
  public Object resolveArgument(...) throws Exception {
    Class expectedClass = parameter.getParameterType();
    MediaType expectedMediaType = computeFrom(webRequest);
    for(HttpMessageConverter converter : httpMessageConverters) {
      if(converter.canRead(expectedClass,expectedMediaType)) {
        return converter.read(expectedClass,webRequest);
      }      
    }
  }

}
 \end{javacode}

\end{frame}

\begin{frame}[fragile]
 \frametitle{\code{HandlerMethodArgumentResolver}}
 \begin{itemize}
  \item Il s'agit bien du principe...
  \item Véritable implémentation : \code{RequestResponseBodyMethodProcessor}
 \end{itemize}
\end{frame}

\begin{frame}[fragile]
 \frametitle{Positionner un \code{HandlerMethodArgumentResolver}}
 
 \begin{xmlcode}
<mvc:annotation-driven>
  <mvc:argument-resolvers>
    <bean class="com.zenika.MyHandlerMethodArgumentResolver" />
  </mvc:argument-resolvers>
</mvc:annotation-driven>
 \end{xmlcode}

\end{frame}

\begin{frame}[fragile]
 \frametitle{Et à la sortie ?}
 
 \begin{itemize}
  \item Même extensibilité pour l'objet retourné par un contrôleur
 \end{itemize}
 
 \begin{javacode}
@RequestMapping(value="/contacts/{id}",method=RequestMethod.GET)
public ResponseEntity<Contact> contact(@PathVariable Long id) {	
  Contact contact = contactRepository.findOne(id);
  ResponseEntity<Contact> response = new ResponseEntity<Contact>(
    contact,
    contact == null ? HttpStatus.NOT_FOUND : HttpStatus.OK
  );
  return response;
}
 \end{javacode}

\end{frame}

\begin{frame}[fragile]
 \frametitle{\code{HandlerMethodReturnValueHandler}}
 
 \begin{javacode}
public interface HandlerMethodReturnValueHandler {

  boolean supportsReturnType(MethodParameter returnType);

  void handleReturnValue(Object returnValue,
      MethodParameter returnType,
      ModelAndViewContainer mavContainer,
      NativeWebRequest webRequest) throws Exception;

}
 \end{javacode}

\end{frame}