\section{Repository custom avec Spring Data JPA}

\begin{frame}
 \frametitle{Outline}
 \scriptsize{\tableofcontents[currentsection,currentsubsection]}
\end{frame}


\begin{frame}
 \frametitle{Pourquoi un repository custom ?}
 \begin{itemize}
  \item Quand les méthodes CRUD ne suffisent pas
  \item Quand les méthodes par convention de nommage ne suffisent pas
  \item On peut alors implémenter notre propre repository
  \item Spring Data JPA le combinera avec un repository dynamique
 \end{itemize}
\end{frame}

\begin{frame}[fragile]
 \frametitle{Déclaration des interfaces}
 \begin{javacode}
public interface ContactRepositoryCustom {

  List<Contact> findByExample(Contact contact);

}

public interface ContactRepository extends ContactRepositoryCustom, 
                                           Repository<Contact,Long> {

  List<Contact> findByLastname(String lastname);

}
 \end{javacode}
\end{frame}

\begin{frame}[fragile]
 \frametitle{Implémentation custom}
 \begin{javacode}
public class ContactRepositoryImpl implements ContactRepositoryCustom {

  @PersistenceContext private EntityManager em;

  @Override
  public List<Contact> findByExample(Contact contact) {
    (...)
  }

}
 \end{javacode}
\end{frame}

\begin{frame}
 \frametitle{Conventions}
 \begin{itemize}
  \item Interfaces et implémentation dans le même package
  \item Classe d'implémentation doit finir par \code{Impl}
  \item Le repository est détecté et combiné automatiquement
 \end{itemize}
\end{frame}

\begin{frame}[fragile]
 \frametitle{Si le postfix par défaut ne convient pas}
 \begin{itemize}
  \item Postfix \code{Jpa} plutôt que \code{Impl}
 \end{itemize}
 \begin{xmlcode}
<jpa:repositories base-package="com.zenika.nordnet.repository" 
                  repository-impl-postfix="Jpa" /> 
 \end{xmlcode}
\end{frame}

\begin{frame}[fragile]
 \frametitle{Travailler repository par repository}
 \begin{itemize}
  \item Réglage avec \code{<jpa:repository />}
 \end{itemize}
 \begin{xmlcode}
<jpa:repositories base-package="com.zenika.nordnet.repository">
  <jpa:repository id="contactRepository" 
                  repository-impl-postfix="Jpa"/>
</jpa:repositories>
 \end{xmlcode}
\end{frame}

\begin{frame}[fragile]
 \frametitle{Portée de \code{<jpa:repository />}}
 \begin{itemize}
  \item \code{<jpa:repository />} désactive la détection automatique des implémentations custom
  \item Tous les repositories custom doivent être déclarés explicitement
 \end{itemize}
 \begin{xmlcode}
<jpa:repositories base-package="com.zenika.nordnet.repository">
  <jpa:repository id="contactRepository" 
                  repository-impl-postfix="Jpa" />
  <jpa:repository id="addressRepository" 
                  custom-impl-ref="addressRepositoryCustom" />
</jpa:repositories>

<bean id="addressRepositoryCustom" 
      class="c.z.nordnet.repository.JpaAddressRepositoryCustom" />
 \end{xmlcode}
\end{frame}