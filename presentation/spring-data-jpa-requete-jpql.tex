\section{JPQL avec Spring Data JPA}

\begin{frame}
 \frametitle{Outline}
 \scriptsize{\tableofcontents[currentsection,currentsubsection]}
\end{frame}


\begin{frame}
 \frametitle{Pourquoi JPQL avec les repositories CRUD}
 \begin{itemize}
  \item Quand les méthodes CRUD ne suffisent pas
  \item Quand les méthodes par convention de nommage ne suffisent pas
  \item Quand une implémentation custom ne se justifie pas
  \begin{itemize}
    \item Aucun traitement nécessaire...
    \item ... juste exécution de la requête et récupération du résultat
  \end{itemize}
 \end{itemize}
\end{frame}

\begin{frame}[fragile]
 \frametitle{\code{@Query}}
 \begin{javacode}
public interface ContactRepository extends Repository<Contact,Long> {

  @Query("from Contact c where c.lastname = ?1")
  List<Contact> findByLastname(String lastname);
	
}
 \end{javacode}
\end{frame}

\begin{frame}[fragile]
 \frametitle{Binging des paramètres avec \code{@Param}}
 \begin{itemize}
  \item Par défaut : paramètres JPQL/Java bindés selon leur index
  \item \code{@Param} permet de binder par nom
 \end{itemize}

 \begin{javacode}
public interface ContactRepository extends Repository<Contact,Long> {

  @Query(
    "from Contact c "+
    "where c.firstname = :firstname and c.lastname = :lastname"
  )
  Contact findByFirstnameAndLastname(
     @Param("firstname") String firstname, 
     @Param("lastname") String lastname);

}
 \end{javacode}
\end{frame}

\begin{frame}[fragile]
 \frametitle{Requête nommée}
 \begin{itemize}
  \item Directement sur l'entité
  \item Convention de nommage dans l'interface
 \end{itemize}
 \begin{javacode}
@Entity
@NamedQuery(
  name = "Contact.findByLastname",
  query = "from Contact c where c.lastname = ?1"
)
public class Contact { (...) }

public interface ContactRepository extends Repository<Contact,Long> {

  List<Contact> findByLastname(String lastname);

}
 \end{javacode}

\end{frame}

\begin{frame}[fragile]
 \frametitle{Requête de modification avec \code{@Modifying}}
  \begin{javacode}
public interface ContactRepository extends Repository<Contact,Long> {

  @Modifying
  @Query("update Contact c set c.lastname = ?1 where c.lastname = ?2")
  @Transactional
  void setNewLastname(String newLastname,String oldLastname);

}
 \end{javacode}
 \begin{itemize}
  \item \code{EntityManager} vidé après l'exécution
  \begin{itemize}
    \item Car des entités peuvent être désynchronisées
    \item Attribut \code{clearAutomatically} disponible pour choisir
  \end{itemize}
  \item Ne pas oublier \code{@Transactional} !
 \end{itemize}
\end{frame}