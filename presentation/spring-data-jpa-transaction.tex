\section{Transaction}

\begin{frame}
 \frametitle{Outline}
 \scriptsize{\tableofcontents[currentsection,currentsubsection]}
\end{frame}


\begin{frame}
 \frametitle{Rappel : transaction avec Spring}
 \begin{itemize}
  \item Spring gère les transactions déclarativement
  \item Réglage par XML ou par annotation
  \item Spring propose aussi une API et un \code{TransactionTemplate}
  \begin{itemize}
   \item Rarement utilisé dans les applications
   \item Pratique quand la démarcation déclarative est insuffisante
  \end{itemize}
 \end{itemize}
\end{frame}


\begin{frame}[fragile]
 \frametitle{Rappel : transaction avec Spring}

 \begin{itemize}
  \item Configuration du \code{PlatformTransactionManager}
  \item Activation de la configuration par annotation
 \end{itemize}

 \begin{xmlcode}
<bean id="transactionManager" 
      class="org.springframework.orm.jpa.JpaTransactionManager">
  <property name="entityManagerFactory" ref="entityManagerFactory" />
</bean>

<tx:annotation-driven />
 \end{xmlcode}
\end{frame}

\begin{frame}
 \frametitle{Rappel : transaction avec Spring}

 \begin{itemize}
  \item D'autres implémentations de \code{PlatformTransactionManager}
  \begin{itemize}
   \item Pour Hibernate, JDBC, JTA, etc.
  \end{itemize}
  \item \code{transactionManager} est le nom habituel du bean
  \item Bien ajouter le namespace \code{tx}
 \end{itemize}
 
\end{frame}

\begin{frame}[fragile]
 \frametitle{Rappel : transaction avec Spring}

 \begin{itemize}
  \item Annoter les classes avec \code{@Transactional}
  \begin{itemize}
   \item Possibilité d'annoter les interfaces ou les classes parentes
  \end{itemize}
 \end{itemize}
 
 \begin{javacode}
@Transactional
public class ContactServiceImpl implements ContactService {

  public void delete(Contact ... contacts) { ... }

}
 \end{javacode}
 
\end{frame}

\begin{frame}[fragile]
 \frametitle{Rappel : transaction avec Spring}
 
 \begin{javacode}
contactService.delete(contact1,contact2);
 \end{javacode}

 \begin{itemize}
  \item L'appel de méthode est intercepté
  \item Une transaction est démarrée
  \item Le code applicatif est appelé
  \item La transaction est committée
  \item La méthode retourne
  \item Si une \code{RuntimeException} est lancée, la transaction est annulée
 \end{itemize}
 
\end{frame}

\begin{frame}
 \frametitle{Spring Data JPA et les transactions}

 \begin{itemize}
  \item Les méthodes des repositories sont transactionnelles
  \item Les méthodes de sélection sont en lecture seule
   \begin{itemize}
    \item \code{findOne}, \code{findAll}, etc.
   \end{itemize}
  \item Les méthodes de modification ne sont pas en lecture seule
   \begin{itemize}
    \item \code{save}, \code{delete}, etc.
   \end{itemize}
  \item Les méthodes ajoutées sont en lecture seule !
   \begin{itemize}
    \item \code{@Query}, convention, implémentations custom
   \end{itemize}
  \item Pour les méthodes ajoutées et modifiant des données, apposer
   \begin{itemize}
    \item \code{@Modifying}
    \item \code{@Transactional} 
   \end{itemize}
 \end{itemize}
 
\end{frame}

\begin{frame}[fragile]
 \frametitle{\code{SimpleJpaRepository}}

 \begin{itemize}
  \item Implémentation utilisée par défaut
  \item Jetons un coup d'oeil...
 \end{itemize}

 \begin{javacode}
@org.springframework.stereotype.Repository
@Transactional(readOnly = true)
public class SimpleJpaRepository<T, ID extends Serializable> 
       implements JpaRepository<T, ID>, JpaSpecificationExecutor<T> {
       
  public T findOne(ID id) { (...) }

  @Transactional
  public void deleteAll() { (...) }
  
  (...)
       
}
 \end{javacode}
 
\end{frame}