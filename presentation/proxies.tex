\section{Proxies}

\begin{frame}
 \frametitle{Outline}
 \scriptsize{\tableofcontents[currentsection,currentsubsection]}
\end{frame}


\begin{frame}
 \frametitle{Proxies dans Spring}
 \begin{itemize}
  \item Principe très utilisé dans Spring
  \item Un mécanisme pour implémenter la programmation orientée aspect
  \item Permet d'ajouter du comportement de façon transparente
  \item Ex. : begin/commit de transaction, sécurité, logging
 \end{itemize}

\end{frame}

\begin{frame}[fragile]
 \frametitle{Principe d'un proxy (AOP)}
 \begin{javacode}
OrderConfirmation o = orderService.submit(order);
 \end{javacode}

TODO mettre une image

\end{frame}

\begin{frame}
 \frametitle{Proxies dans Spring Data JPA}
 \begin{itemize}
  \item Principe légèrement différent des proxies AOP
  \item Il n'y pas de cible
  \item Le comportement de la méthode est implémenté à la volée
  \item Ex. de critères d'implémentation : le nom de la méthode
 \end{itemize}

\end{frame}

\begin{frame}
 \frametitle{Proxy JDK}
 \begin{itemize}
  \item Spring utilise les proxies JDK
  \item L'API proxies JDK est simple mais bas niveau
  \item Spring l'enrichit, notamment via AspectJ
  \item Ex. : notion de poincut, configuration par annotations, etc.
 \end{itemize}
\end{frame}

\begin{frame}[fragile]
 \frametitle{Créer un proxy JDK}

\begin{javacode}
ContactRepository repo = (ContactRepository) Proxy.newProxyInstance(
  getClass().getClassLoader(),
  new Class<?>[]{ContactRepository.class},
  invocationHandler
);
\end{javacode}

 \begin{itemize}
  \item Qu'est-ce que l'\code{InvocationHandler} ?
 \end{itemize}

\end{frame}

\begin{frame}[fragile]
 \frametitle{\code{InvocationHandler}}
 \begin{itemize}
  \item L'\code{InvocationHandler} implémente le comportement
 \end{itemize}

\begin{javacode}
InvocationHandler invocationHandler = new InvocationHandler() {      
  public Object invoke(Object proxy, Method method, Object[] args)
                                throws Throwable {
    System.out.println(method.getName());
    return null;
  }
};
ContactRepository repo = (ContactRepository) Proxy.newProxyInstance(..);
repo.findAll(); // affiche ``findAll''
\end{javacode}

\end{frame}

\begin{frame}[fragile]
 \frametitle{Proxies dans Spring Data JPA}
 \begin{itemize}
  \item Un repository avec un interface custom
  \item Le proxy correspond ``route'' les appels :
  \begin{itemize}
   \item Implémentation par défaut pour les méthodes CRUD
   \item Implémentation custom pour les méthodes de l'interface custom
   \item Implémentation suivant l'annotation apposée
   \item Implémentation dynamique pour les méthodes basées sur les conventions
   \item etc.
  \end{itemize}
 \end{itemize}

\end{frame}