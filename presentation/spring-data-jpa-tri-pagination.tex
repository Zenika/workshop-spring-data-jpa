\section{Tri et pagination}

\begin{frame}
 \frametitle{Outline}
 \scriptsize{\tableofcontents[currentsection,currentsubsection]}
\end{frame}


\begin{frame}
 \frametitle{Principes}
 \begin{itemize}
  \item Spring Data JPA peut trier et paginer des données
  \item Il suffit de rajouter des paramètres dans les méthodes
  \item Rajoute de la flexibilité aux méthodes
  \begin{itemize}
   \item Pas d'instruction de tri en dur dans les requêtes JPQL
  \end{itemize}
 \end{itemize}

\end{frame}

\begin{frame}[fragile]
 \frametitle{Tri avec \code{Sort}}
 \begin{itemize}
  \item Ajouter un paramètre de type \code{Sort} à la méthode
 \end{itemize}
 \begin{javacode}
public interface ContactRepository extends Repository<Contact,Long> {

  List<Contact> findByLastname(String lastname, Sort sort);

}
 \end{javacode}
\end{frame}

\begin{frame}[fragile]
 \frametitle{Tri avec \code{Sort}}
 \begin{itemize}
  \item Lors de l'appel, spécifier
  \begin{itemize}
   \item Un ordre avec l'énumération \code{Direction}, \code{ASC} ou \code{DESC}
   \item Une ou des propriétés
  \end{itemize}
 \end{itemize}
 \begin{javacode}
repo.findByLastname("Dalton",new Sort(Direction.DESC,"firstname"));
repo.findByLastname("Dalton",new Sort("firstname")); // ASC par d\'efaut
 \end{javacode}
\end{frame}

\begin{frame}[fragile]
 \frametitle{Pagination avec \code{Pageable}}
 \begin{itemize}
  \item Ajouter un paramètre de type \code{Pageable} à la méthode
 \end{itemize}
 \begin{javacode}
public interface ContactRepository extends Repository<Contact,Long> {

  List<Contact> findByLastname(String lastname, Pageable pageable);

}
 \end{javacode}
\end{frame}

\begin{frame}[fragile]
 \frametitle{Pagination avec \code{Pageable}}
 \begin{itemize}
  \item Utiliser \code{PageRequest} comme implémentation de \code{Pageable}
  \item Nécessité de passer un \code{Sort} pour des pages cohérentes
 \end{itemize}
 \begin{javacode}
PageRequest pageable = new PageRequest(0, 10,new Sort("firstname"));
List<Contact> contacts = repo.findByLastname("Dalton",pageable);
 \end{javacode}
\end{frame}

\begin{frame}[fragile]
 \frametitle{Récupérer une \code{Page}}
 \begin{itemize}
  \item Retourner une \code{Page<T>}
  \item Donne des informations sur les données et leur pagination
  \item Implique une requête \code{count} à chaque appel
 \end{itemize}
 \begin{javacode}
public interface ContactRepository extends Repository<Contact,Long> {

  Page<Contact> findByLastname(String lastname, Pageable pageable);

}
 \end{javacode}
\end{frame}

\begin{frame}[fragile]
 \frametitle{Exploiter la \code{Page}}
 \begin{itemize}
  \item Informations disponibles :
  \begin{itemize}
   \item Les enregistrements !
   \item Numéro de la page
   \item Nombre d'éléments total
   \item Première page ou dernière page
   \item etc.
  \end{itemize}
 \end{itemize}
 \begin{javacode}
PageRequest pageable = new PageRequest(0, 10,new Sort("firstname"));
Page<Contact> page = repo.findByLastname("Dalton",pageable);
List<Contact> contacts = page.getContent();
if(page.isFirstPage()) { (...) }
 \end{javacode}
\end{frame}

\begin{frame}[fragile]
 \frametitle{\code{Sort}, \code{Page} et \code{Query}}
 \begin{itemize}
  \item \code{Query} peut cohabiter avec \code{Tri} et \code{Page}
 \end{itemize}
 \begin{javacode}
public interface ContactRepository extends Repository<Contact,Long> {

  @Query("from Contact c join fetch a.address where c.firstname = ?1")
  List<Contact> findByLastname(String lastname, Sort sort);
  
}
 \end{javacode}
\end{frame}

\begin{frame}[fragile]
 \frametitle{\code{Page} et \code{Query}}
 \begin{itemize}
  \item Possibilité de préciser sa requête de \code{count}
 \end{itemize}
 \begin{javacode}
public interface ContactRepository extends Repository<Contact,Long> {

  @Query(
    value="from Contact c join fetch a.address where c.firstname = ?1",
    countQuery="select count(*) from Contact c where c.firstname = ?1"
  )
  Page<Contact> findByLastname(String lastname, Pageable pageable);
  
}
 \end{javacode}
\end{frame}